\documentclass[11pt,a4paper]{moderncv}

% moderncv themes
%\moderncvtheme[blue]{casual}                 % optional argument are 'blue' (default), 'orange', 'red', 'green', 'grey' and 'roman' (for roman fonts, instead of sans serif fonts)
\moderncvtheme[green]{classic}                % idem

\usepackage[T1]{fontenc}
% character encoding
\usepackage[utf8x]{inputenc}                   % replace by the encoding you are using
\usepackage[italian]{babel}

% adjust the page margins
\usepackage[scale=0.8]{geometry}
\recomputelengths                             % required when changes are made to page layout lengths

\fancyfoot{} % clear all footer fields
\fancyfoot[LE,RO]{\thepage}           % page number in "outer" position of footer line
\fancyfoot[RE,LO]{\footnotesize } % other info in "inner" position of footer line

% personal data
\firstname{Varsha}
\familyname{\\ \\Kirani Gopinath}
\title{Curriculum Vitae}               % optional, remove the line if not wanted
%\address{Våxtorpsgränd 10, Älvsjö}{12573 Stockholm ,Sweden}    % optional, remove the line if not wanted
%\mobile{+46729347913}                    % optional, remove the line if not wanted
%\phone{+46729347913}                      % optional, remove the line if not wanted
%\fax{312 996 1491}                          % optional, remove the line if not wanted
%\email{varsha@kth.se}                      % optional, remove the line if not wanted
%\extrainfo{additional information (optional)} % optional, remove the line if not wanted
\photo[84pt]{varsha.jpg}          % '64pt' is the height the picture must be resized to and 'picture' is the name of the picture file; optional, remove the line if not wanted

\quote{"All that we are is the result of what we have thought." -- Buddha}

%----------------------------------------------------------------------------------
%            content
%----------------------------------------------------------------------------------
\begin{document}
\maketitle

\vspace{-10mm}

%Section
\section{Info}
\cvitem{Born}{\small June 4th, 1991 at Sira,Tumkur (India)\normalsize}
\cvitem{Address}{Våxtorpsgränd 10, Älvsjö, 12573 Stockholm ,Sweden}
\cvitem{Phone}{+46 729 347 913}
\cvitem{e-mail}{varsha@kth.se}

%Section
%\section{Education}
%\cventry{2014 - present}{Master of Science in Machine Learning}{\href{https://www.kth.se/en/studies/master/kth/machinelearning/description-1.48533}{KTH Royal Institute Of Technology}}{Stockholm}{Sweden}{ 2 year programme, majoring in Machine Learning and Computer Vision. Other important courses include: Artificial Neural Networks, Artificial Intelligence and Information Retrieval.}
%\cventry{2010 - 2013}{Bachelor of Science in Computer Science Engineering}{\href{http://www.msrit.edu/}{M.S.Ramaiah Institute Of Technology}}{Bangalore}{India}{4 year programme majoring in algorithms, computer architectures and basics of programming languages and platforms. Other courses included part of this program were Graphics and Visualisation, Discrete Mathematics, Database Systems and Operating Systems}

\section{Education}
\cvline{2014 - present}{\textbf{Master of Science in Machine Learning}}
\cvline{}{\href{https://www.kth.se/en/studies/master/kth/machinelearning/description-1.48533}{KTH Royal Institute Of Technology},Stockholm, Sweden}
\cvline{}{2 year programme, majoring in Machine Learning and Computer Vision. Other important courses include: Artificial Neural Networks, Artificial Intelligence and Information Retrieval.}

\cvline{2010 - 2013}{\textbf{Bachelor of Science in Computer Science Engineering}}
\cvline{}{\href{http://www.msrit.edu/}{M.S.Ramaiah Institute Of Technology},Bangalore, India}
\cvline{}{4 year programme majoring in algorithms, computer architectures and basics of programming languages and platforms. Other courses included part of this program were Graphics and Visualisation, Discrete Mathematics, Database Systems and Operating Systems}

\section{Technical Skills}
\cvitem{Languages}{C++, JAVA, PYTHON, SQL, MySQL, SQLITE Latex, Javascript, HTML5, CSS3, Jquery, C, C\#}
\cvitem{Tools}{Matlab, OpenGL, Visual Studio, Eclipse IDE, Netbeans IDE}

\section{Recent Coursework}
\cvlistitem{Artificial Intelligence}
\cvlistitem{Machine Learning-Advanced}
\cvlistitem{Artificial Neural Network}
\cvlistitem{Search Engine and Information Retrieval}
\cvlistitem{Applied Bioinformatics}
\cvlistitem{Image Based Recognition and Classification (May 2015)}
\cvlistitem{Visualisation (May 2015)}

\pagebreak
%\section{Academic Interests}
%\cvlistitem{Search Engine and Information Retrieval}
%\cvlistitem{Visualisation}
%\cvlistitem{Machine Learning}
%\cvlistitem{Data Analytics and data mining}
%\cvlistitem{Web Design}

\section{Projects and Abstracts}
\vspace{-3mm}

\subsection{Classifying Proteins based on Signal Peptides using Hidden Markov Models}
\cvitem{Applied Bioinformatics}{\textit{Signal peptides} are vital for many cell functionalities. Our technique was built to classify all the proteins as containing such signal peptides or not, based on their sequence of molecular structures. Real world data was from \textit{Ensembl's BioMart} service. During training, each protein is modelled into a Hidden Markov Model(HMM) and labelled according to the presence/absence of signal peptides. The new unseen protein is classified using \textit{Forward Algorithm}. I also used BioPython to extract the data into required format. (1 member team)}

\subsection{Hidden Markov Model based Video Game Agent for Duck Hunt}
\cvitem{Artificial Intelligence}{The agent playing the game is required to \textit{shoot} birds of different species flying in the sky in characteristic patterns. This entailed to build an agent that trained HMMs and used them to predict the next move of the birds and classify them based on its species. I implemented the Baum-Welch Algorithm for training HMMs and used the forward algorithm for classification and prediction. (1 member team)}

\subsection{Natural Language Generator for Wardrobe Tips depending on Weather}
\cvitem{Artificial Intelligence}{We used context free grammar (CFG) and a large raw corpus to build parse-trees which were then organised into tree-banks. Such tree-banks could then be used as probabilistic CFG (PCFG). Different generators (Greedy, Viterbi, n-gram) tapped the PCFG conditionally and produced sentences or phrases as wardrobe tips. Evaluation of the application was done using cross validation method where \textit{string-edit} and \textit{BLEU methods} are used. A human survey was also made to serve as an informal Turing-Test. (4 member team)}


\subsection{Improving Elevator Performance Using Reinforcement Learning}
\cvitem{Machine Learning} {The challenge was to learn an optimum elevator operation policy from experience, in order to reduce the travel overhead and improve passenger service. We constructed an Artificial Neural Network to learn a map from the passenger demand scenario to the action sequence an elevator must take. 
Elevator simulation was implemented to represent real time scenario. Different kinds of events (like passenger arrival, elevator arrival, passenger departure, elevator moving towards particular floor etc) can occur at different times. Thus the simulation lies in continuous state space. Q-learning which is a model free reinforcement learning technique is used to find optimal action for the particular state. Q-learning requires state to state transition probabilities in order to decide upon the action to choose. Since the elevator system lies in continuous state space, its impossible to maintain lookup table. Instead an artificial neural network is used and the weights of it is updated using back propagation algorithm.
We also, simulated an elevator-calling system including a visualisation. (5 member team)}

\subsection{Artificial Neural Network to Solve Travelling Salesmen Problem}
\cvitem{Artificial Neural Network}{The challenge was to find the shortest path to travel through the given number of cities. This was solved using \textit{Self Organizing Maps} (SOM) which is a type of artificial neural networks. SOM algorithm was analyzed for various values of parameters, like neuron number, epochs, learning rate etc. (2 member team)}

\subsection{Comparison of Different Methods for Search Engines}
\cvitem{Search Engine and Information Retrieval}{For \textit{DavisWiki} corpus, a search engine was built where one can search for documents which has main searching facilities like intersection query, phrase query, tf-idf ranked retrieval , page rank retrieval and combination of any two. The main intention was to play around and learn different methods in search engine and information retrieval. (1 member team)}

\subsection{Building Node Intelligence in MANETs for Efficient Routing}
\cvitem{Bachelor Thesis}{As single protocol will not serve all the purposes required for communication in Mobile Adhoc NETworks (MANETs), switching between protocols can resolve this problem but builds up an overhead in the network as it needs an external agent to monitor the environment and dictate to the network which protocol should be chosen. The nodes present in the network can be made intelligent enough to sense its environment and select the suitable protocol, that is, the switching can be done at the node level so as to reduce the overhead in the network and to make the routing more efficient. The method learns which protocol is a better fit for different scenarios using cost metrics such as node density in the network and message queue time. (4 member team)}

\subsection{Train systems using JAVA and SQL}
\cvitem{Programming}{A simple application of train schedule systems where both passengers and admin were able to get information about the trains like arrival time, departure time,complete route map and other info were made available. The main intention of the project was to explore JAVA functionalities and Database Management Systems. (2 member team)}

\subsection{Jumble Mumble, the word game in C\#}
\cvitem{Programming}{It is a simple word game where user is given jumbled letters and asked to form multiple words. Scores were given according to length of the word and also weights on each alphabet. The whole application was done using C\# and .Net assemblies using Visual Studio.(2 member team)}
%Section
\section{Publications}
\vspace{-3mm}
\subsection{Building Node Intelligence in MANETs for Efficient Routing}
\cvline{}{International Journal of Computer Science and Technology (IJCST)}
\cvline{}{Year of Publication : January 2014 - March 2014}
\cvline{}{Volume 5, SPL1}
\cvline{}{PaperID - IJCST/5/Spl1/CC1114}
%\cventry{year--year}{degree or job title}{institution or employer}{city}{grade}{description}
\section{Industry Experiences}
\vspace{-2mm}

\subsection{Work}
\cventry{2013--2014}{Associate Software Developer}{SAP Labs India}{Visualization Recommendation Team}{Bangalore}{I worked in the \textit{Business Intelligence Team} at SAP Labs, India - contributing to the development of \textit{Lumira}. \textit{Lumira} is a data analysis tool which provides efficient visualisations of large data focussed on businesses and economics. The main scientific challenge was to recommend the best visualisations depending on the data provided. I learnt to code in the Model-View-Controller software pattern of coding. Working in a corporate environment has also improved my teamwork skills and discussion-for-consensus capabilities.}
\subsection{Internship}
\cventry{Feb 2013 - Jun 2013}{ABAP-Language programmer}{SAP Labs, India}{Team- Student Life Cycle Management}{Bangalore}{I worked on a database product for Student-Life-Cycle-Management. Vital technical challenges were the understanding  of the workflow of the entire product and growing agility in using Advanced Business Application Programming (ABAP) language.}

%%Section
%\section{Interests and Hobbies}
%\cvline{}{I love any sport but I prefer Badminton and Table Tennis. I love social gathering and interested in spending time with people more. Love gaming and a big time NARUTO fan.}

\end{document}