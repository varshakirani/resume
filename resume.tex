\documentclass[11pt,a4paper]{moderncv}

% moderncv themes
%\moderncvtheme[blue]{casual}                 % optional argument are 'blue' (default), 'orange', 'red', 'green', 'grey' and 'roman' (for roman fonts, instead of sans serif fonts)
\moderncvtheme[green]{classic}                % idem

\usepackage[T1]{fontenc}
% character encoding
\usepackage[utf8x]{inputenc}                   % replace by the encoding you are using
\usepackage[italian]{babel}

% adjust the page margins
\usepackage[scale=0.8]{geometry}
\recomputelengths                             % required when changes are made to page layout lengths

\fancyfoot{} % clear all footer fields
\fancyfoot[LE,RO]{\thepage}           % page number in "outer" position of footer line
\fancyfoot[RE,LO]{\footnotesize } % other info in "inner" position of footer line

% personal data
\firstname{Varsha}
\familyname{\\ \\Kirani Gopinath}
\title{Curriculum Vitae}               % optional, remove the line if not wanted
%\address{Våxtorpsgränd 10, Älvsjö}{12573 Stockholm ,Sweden}    % optional, remove the line if not wanted
%\mobile{+46729347913}                    % optional, remove the line if not wanted
%\phone{+46729347913}                      % optional, remove the line if not wanted
%\fax{312 996 1491}                          % optional, remove the line if not wanted
%\email{varsha@kth.se}                      % optional, remove the line if not wanted
%\extrainfo{additional information (optional)} % optional, remove the line if not wanted
\photo[84pt]{varsha.jpg}          % '64pt' is the height the picture must be resized to and 'picture' is the name of the picture file; optional, remove the line if not wanted

\quote{"All that we are is the result of what we have thought." -- Buddha}

%----------------------------------------------------------------------------------
%            content
%----------------------------------------------------------------------------------
\begin{document}
\maketitle

\vspace{-10mm}

%Section
\section{Info}
\cvitem{Born}{\small June 4th, 1991 at Sira,Tumkur (India)\normalsize}
\cvitem{Address}{Våxtorpsgränd 10, Älvsjö, 12573 Stockholm ,Sweden}
\cvitem{Phone}{+46 729 347 913}
\cvitem{e-mail}{varsha@kth.se}

%Section
\section{Education}
\cventry{2014 - present}{Master of Science in Machine Learning}{\href{https://www.kth.se/en/studies/master/kth/machinelearning/description-1.48533}{KTH Royal Institute Of Technology}}{Stockholm}{Sweden}{ 2 year programme, majoring in Machine Learning and Computer Vision. Other important courses include: Artificial Neural Networks, Artificial Intelligence and Information Retrieval.}
\cventry{2010 - 2013}{Bachelor of Science in Computer Science Engineering}{\href{http://www.msrit.edu/}{M.S.Ramaiah Institute Of Technology}}{Bangalore}{India}{4 year programme majoring in algorithms, computer architectures and basics of programming languages and platforms. Other courses included part of this program were Graphics and Visualisation, Discrete Mathematics, Database Systems and Operating Systems}


\section{Technical Skills}
\cvitem{Languages}{C++, JAVA, PYTHON, SQL, MySQL, SQLITE Latex, Javascript, HTML5, CSS3, Jquery, C, C\#}
\cvitem{Tools}{Matlab, OpenGL, Visual Studio, Eclipse IDE, Netbeans IDE}

\section{Recent Coursework}
\cvlistitem{Artificial Intelligence}
\cvlistitem{Machine Learning-Advanced}
\cvlistitem{Artificial Neural Network}
\cvlistitem{Search Engine and Information Retrieval}
\cvlistitem{Applied Bioinformatics}
\cvlistitem{Image Based Recognition and Classification (May 2015)}
\cvlistitem{Visualisation (May 2015)}

\pagebreak
%\section{Academic Interests}
%\cvlistitem{Search Engine and Information Retrieval}
%\cvlistitem{Visualisation}
%\cvlistitem{Machine Learning}
%\cvlistitem{Data Analytics and data mining}
%\cvlistitem{Web Design}

\section{Projects and Abstracts}
\vspace{-3mm}
\subsection{Hidden Markov Model based Video Game Agent for Duck Hunt}
\cvitem{Artificial Intelligence}{The agent playing the game is required to \textit{shoot} birds of different species flying in the sky in characteristic patterns. This entailed to build an agent that trained HMMs and used them to predict the next move of the birds and classify them based on its species. I implemented the Baum-Welch Algorithm for training HMMs and used the forward algorithm for classification and prediction. (1 member team)}

\subsection{Improving Elevator Performance Using Reinforcement Learning}
\cvitem{Machine Learning} {The challenge was to learn an optimum elevator operation policy from experience, in order to reduce the travel overhead and improve passenger service. We constructed an Artificial Neural Network to learn a map from the passenger demand scenario to the action sequence an elevator must take. Different events in the simulation were in continuous time space. Because of huge branching factor, instead of look-up table, Q learning and neural networks were used to learn the system (conventionally state transition probabilities will be maintained). The Q-learning will back-propagate the error into the neural network to update the weights.Total waiting time of the passenger is reinforcement signal. Whenever an event occurs, reinforcement signal is used in Q-learning. We also, simulated an elevator-calling system including a visualisation. (5 member team)}

\subsection{Classifying Proteins based on Signal Peptides using Hidden Markov Models}
\cvitem{Applied Bioinformatics}{The goal was to predict whether the protein contains signal peptide or not which is used in localization of proteins in the cell to do their job properly. This was supervised learning where the training data is real world data extracted from Ensembl's BioMart service. Each protein from training data is represented as Hidden Markov Model(HMM) and  all the HMMs were converged using Baum Welch Algorithm by maintaining the proteins in two clusters(with and without signal peptide). The new unseen protein is classified using Forward Algorithm. I also used BioPython to extract the data into required format. (1 member team)}

\subsection{Comparison of different methods for Search Engine of DavisWiki }
\cvitem{Search Engine and Information Retrieval}{For DavisWiki corpus, a search engine was built where one can search for documents which has main searching facilities like intersection query, phrase query, tf-idf ranked retrieval , page rank retrieval and combination of any two. The main intention was to learn different methods in search engine and information retrieval.}
\subsection{Generating Wardrobe Tips depending on Weather using Natural Language Generator}
\cvitem{Artificial Intelligence}{The aim was to build an application which gives wardrobe tips according to weather condition. We developed treebank trained statistical generator using context free grammar(CFG) and raw corpus to create probabilistic CFG (PCFG). Using this PCFG, different generators (Greedy, Viterbi,n-gram) produces sentences or phrases(wardrobe tip). We also implemented naive generator called probabilistic generator where sentences are generated according to probabilities of rules in PCFG at every step of treebank. Evaluation of the application was done using cross validation method where String-edit and BLEU methods are used. Human survey was also taken. (4 member team)  }
\subsection{Self Organizing Map to solve Traveling Salesmen Problem using Artificial neural network }
\cvitem{Artificial Neural Network}{The challenge was to find the shortest path for the given number of cities. This was done using self organizing map (SOM) algorithm using artificial neural network. For various values of parameters (like neuron number, epochs, learning rate etc), SOM algorithm is analyzed. }
\subsection{Train systems using JAVA and SQL}
\cvitem{Programming}{A simple application of train system where both passengers and admin were able to get information about the trains like arrival time, departure time,complete route map and other info were made available. The main intention of the project was to explore JAVA functionalities and Database Management Systems.(2 member team)	}
\subsection{Jumble Mumble, the word game in C\#}
\cvitem{Programming}{It is a simple word game where user is given jumbled letters and asked to form multiple words. Scores were given according to length of the word and also weights on each alphabet. The whole application was done using C\# and .Net assemblies using Visual Studio.(2 member team)}
\subsection{Building Node Intelligence in MANETs for Efficient Routing}
\cvitem{Bachelor Thesis}{As single protocol will not serve all the purposes required for communication in MANETs, switching between protocols can resolve this problem but builds up an overhead in the network as it need an external agent to monitor the environment and dictate to the network on which protocol should be chosen. Thus,the nodes present in the network can be made intelligent enough to sense its environment and select the suitable protocol ,that is , the switching can be done at the node level itself so as to reduce the overhead in the network and to make the routing more efficient. For real time scenarios and simulations we used NS2.35 simulator. AODV and DSDV protocols are considered. Node density and the pause time of the environment are taken as the cost metrics. At the learning phase, we learn about protocol which better fits at different scenarios. According to this result switching is done at node level to chose the protocol.(4 member team)}

%Section
\section{Publications}
\vspace{-3mm}
\subsection{Building Node Intelligence in MANETs for Efficient Routing}
\cventry{year--year}{degree or job title}{institution or employer}{city}{grade}{description}
\section{Industry Experiences}
\vspace{-2mm}

\subsection{Work}
\cventry{2013--2014}{Associate Software Developer}{SAP Labs India}{Visualization Recommendation Team}{Bangalore}{I worked in the \textit{Business Intelligence Team} at SAP Labs, India - contributing to the development of \textit{Lumira}. \textit{Lumira} is a data analysis tool which provides efficient visualisations of large data focussed on businesses and economics. The main scientific challenge was to recommend the best visualisations depending on the data provided. I learnt to code in the Model-View-Controller software pattern of coding. Working in a corporate environment has also improved my teamwork skills and discussion-for-consensus capabilities.}
\subsection{Internship}
\cventry{Feb 2013 - Jun 2013}{ABAP-Language programmer}{SAP Labs, India}{Team- Student Life Cycle Management}{Bangalore}{I worked on a database product for Student-Life-Cycle-Management. Vital technical challenges were the understanding  of the workflow of the entire product and growing agility in using Advanced Business Application Programming (ABAP) language.}

%Section
\section{Interests and Hobbies}
\cvline{}{\small I love any sport but I prefer Badminton and Table Tennis. I love social gathering and interested in spending time with people more. Love gaming and a big time gay NARUTO fan.}

\end{document}